\section{Macros Provided by this Template}

\subsection{Lists with tighter spacing}

Normal \lstinline{itemize}:

\begin{itemize}
    \item An item
    \item Another item
\end{itemize}

With \lstinline{tight_itemize}:

\begin{tight_itemize}
    \item An item
    \item Another item
\end{tight_itemize}

There is also a \lstinline{tight_enumerate}:

\begin{tight_enumerate}
    \item An item
    \item Another item
\end{tight_enumerate}

There is also a \lstinline{inline_enumerate}:
\begin{inline_enumerate}
    \item An item,
    \item Another item,
        and
    \item Another,
\end{inline_enumerate}
all in a paragraph.


\subsection{Links}

Shortcut to create a \lstinline{mailto:} href, \lstinline{\emailhref}:
\emailhref{example@example.com}


\subsection{TODO Notes}

I have defined a few shortcuts for styled todo notes:

Use \lstinline{towrite} as a placeholder for paragraphs you intend to write.
When you've written it, you can delete the \lstinline{towrite} note.
You can also use \lstinline{outline} for outline markers you want to keep around in draft.

\outline{Idea X}
\towrite{Explain idea X}
\outline{Idea Y}
\towrite{Explain idea Y}


Normal todos are in orange.
\todo{Example todo}

Use \lstinline{ask} to add notes where you need to specifically ask for feedback.
\ask{Question for collaborator}
\ask[Mr. X]{Question for Mr. X}
\ask[][inline]{Question with additional options passed to \lstinline{todo}, e.g. \lstinline{inline}}

Use \lstinline{feedback} to add notes where someone has given you feedback.
\feedback{Some feedback here}
\feedback[Mr. X]{Some feedback from Mr. X}
\feedback[][inline]{Some feedback with additional options passed to \lstinline{todo}, e.g. \lstinline{inline}}



\subsection{TODO-Like Notes that Are Not Listed as Tasks}

These are macros for notes that annotate the draft of the document,
but are not listed as tasks that need to be taken care of.

\draftnote{This is where we demonstrate the draft note}
Use \lstinline{draftnote} for miscellaneous notes.

\guidance{Be sure to include examples of each type here}
Use \lstinline{guidance} for general advice and reminders for this section.

\outline{Example 1}
Use \lstinline{outline} for to mark the outline of the document.



\subsection{Shortcuts for Specific Units of Measurement}

Base-2 data sizes:
\lstinline{kib} (\SI{4}{\kib}),
\lstinline{mib} (\SI{10}{\mib}),
and
\lstinline{gib} (\SI{2}{\gib}).



\subsection{Draft-Only Content}

There is \lstinline{draftonlysection} for an entire section that should not be present in the \lstinline{final} version of the document.
Examples of that are the version information.

Use \lstinline{scratch} to block off text that is on its way to being deleted.
The \lstinline{scratch} macro takes an optional first argument to explain the deletion.

\scratch[This is an example of text that will be omitted in final]{

Imagine this is a previous version of the paragraph.
We are temporarily keeping it around for comparison.

}



\section{Example Graphviz Diagrams}

For an example Graphviz diagram, see
\cref{diagram_example}%
.
The digram was generated from the source given in
\cref{diagram_common,diagram_example_source}%
.

% Captions should go below graphics for images and diagrams.
% HOWEVER they can go above graphs and listings, as a title.
% For proper linking, always put the label AFTER the caption.
\begin{figure}[h]
    \centering
    \includegraphics[width=0.5\textwidth]{diagram_example}
    \caption{Example Graphviz (dot) diagram}
    \label{diagram_example}
\end{figure}

\lstinputlisting[
    caption={M4 Macros included in Graphviz code},
    label=diagram_common,
]{diagram_common.m4.dot}
\lstinputlisting[
    caption={Example Graphviz code},
    label=diagram_example_source,
]{diagram_example.dot}

\section{Common Macros}

\subsection{Lists with tighter spacing}

Normal \lstinline{itemize}:

\begin{itemize}
    \item An item
    \item Another item
\end{itemize}

With \lstinline{tight_itemize}:

\begin{tight_itemize}
    \item An item
    \item Another item
\end{tight_itemize}

There is also a \lstinline{tight_enumerate}:

\begin{tight_enumerate}
    \item An item
    \item Another item
\end{tight_enumerate}


\subsection{Links}

Shortcut to create a \lstinline{mailto:} href, \lstinline{\emailhref}:
\emailhref{example@example.com}


\subsection{TODO Notes}

I have defined a few shortcuts for styled todo notes:

Use \lstinline{towrite} as a placeholder for paragraphs you intend to write.
When you've written it, you can delete the \lstinline{towrite} note,
or change it to \lstinline{written} to leave a record in place.

\towrite{Mention idea X}

\written{Mention idea Y}

Use \lstinline{ask}
\ask{Does this work?}
or \lstinline{askinline} to add notes where you need to specifically ask for feedback.

\askinline{Is this alright?}

Use \lstinline{feedback}
\feedback{Could be punchier?}
or \lstinline{feedbackinline} to add places where someone has given you specific feedback.

\feedbackinline{This is ok}


\subsection{Shortcuts for Specific Units of Measurement}

Base-2 data sizes:
\lstinline{kib} (\SI{4}{\kib}),
\lstinline{mib} (\SI{10}{\mib}),
and
\lstinline{gib} (\SI{2}{\gib}).


\subsection{Draft-Only Content}

There is \lstinline{draftonlysection} for an entire section that should not be present in the \lstinline{final} version of the document.
Examples of that are the version information and the scratchpad.

There is also \lstinline{draftonlynote} for short notes that should disappear that are not TODOs.

\draftonlynote{This will disappear when the document is marked \lstinline{final}.}


\section{Diagrams with Graphviz}

For an example Graphviz diagram, see
\cref{diagram_example}%
.
The digram was generated from the source given in
\cref{diagram_common,diagram_example_source}%
.

% Captions should go below graphics for images and diagrams.
% HOWEVER they can go above graphs and listings, as a title.
% For proper linking, always put the label AFTER the caption.
\begin{figure}[h]
    \centering
    \includegraphics[width=0.5\textwidth]{diagram_example}
    \caption{Example Graphviz (dot) diagram}
    \label{diagram_example}
\end{figure}

\lstinputlisting[
    caption={M4 Macros included in Graphviz code},
    label=diagram_common,
]{diagram_common.m4.dot}
\lstinputlisting[
    caption={Example Graphviz code},
    label=diagram_example_source,
]{diagram_example.dot}
